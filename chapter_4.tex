\chapter{Results and Discussion}

This chapter presents the major outputs of the study, including the construction of the Akeanon text and speech corpora, and the performance evaluation of the developed ASR model.

\section{Constructed Akeanon Text Corpus}

A total of \textbf{25,800} Akeanon words were collected and verified for the text corpus. This collection excludes the Swadesh and SIL word lists and includes a wide variety of root words, derivations, and inflections. Figure~\ref{fig:text-corpus} shows a snapshot of the sheet file that serves as the database of the text corpus.

\begin{figure}[H]
    \centering
    \includegraphics[width=\textwidth]{./figures/text-corpus.png}
    \caption{Snapshot of the Akeanon text corpus}
    \label{fig:text-corpus}
\end{figure}

In addition to the main corpus, the study also translated the Swadesh 207-word list and SIL International's word list into five Akeanon dialects: Standard Akeanon, Bukidnon, Buruangganon, Malaynon, and Nabasnon. Figures~\ref{fig:swadesh-list} and~\ref{fig:sil-list} display sample entries from these translations.

\begin{figure}[H]
    \centering
    \includegraphics[width=\textwidth]{./figures/swadesh.png}
    \caption{Akeanon translations of the Swadesh 207-word list}
    \label{fig:swadesh-list}
\end{figure}

\begin{figure}[H]
    \centering
    \includegraphics[width=\textwidth]{./figures/SIL.png}
    \caption{Akeanon translations of SIL International's word list}
    \label{fig:sil-list}
\end{figure}

The constructed text corpus serves as a foundation for the development of the Akeanon ASR system, providing linguistic diversity and coverage across different dialects.

\section{Constructed Akeanon Speech Corpus}

For the Akeanon speech corpus, \textbf{100} voice recordings were collected, equivalent to about \textbf{16 hours} of raw data. Each recording corresponds to one of the generated text sets and covers various dialects and speaker demographics.

The collected speech data provides the necessary acoustic material for training, validating, and testing the ASR models. The recordings include natural variations in pronunciation, intonation, and pacing, enriching the acoustic modeling phase.

\section{Monophone and Triphone Model Results}

\subsection{Recognition Performance}

The performance of the acoustic models was evaluated using Word Error Rate (WER), which measures the percentage of incorrectly recognized words in the test set. Table~\ref{tab:wer-results} summarizes the WER obtained for each acoustic model.

\begin{table}[H]
    \centering
    \renewcommand{\arraystretch}{1.3}
    \setlength{\tabcolsep}{16pt}
    \caption{Word Error Rate (WER\%) for Each Acoustic Model}
    \label{tab:wer-results}
    \begin{tabular}{|l|c|}
        \hline
        \textbf{Model} & \textbf{WER (\%)} \\
        \hline
        Monophone & 43.64 \\
        Triphone & 6.75 \\
        Triphone + LDA+MLLT & 5.49 \\
        \hline
    \end{tabular}
\end{table}

As shown in the results, increasing model complexity led to improved recognition performance. The monophone model yielded the highest WER, while the triphone model with LDA+MLLT transformations achieved the best result, highlighting the effectiveness of advanced feature modeling techniques.
