%   Filename    : abstract.tex 
\begin{center}
\textbf{Abstract}
\end{center}
\setlength{\parindent}{0pt}
This study aimed to develop foundational resources and acoustic models to support automatic speech recognition (ASR) for the Akeanon language. A text corpus containing \textbf{25,800} verified Akeanon words was constructed, alongside additional translations of the Swadesh 207-word list and SIL International's word list for five major Akeanon dialects. Furthermore, a speech corpus consisting of \textbf{100} voice recordings, totaling to over \textbf{8 hours} of speech data and an additional 31 hours of extracted audio from online resources, was collected to provide training and evaluation material. Using the Kaldi toolkit, ASR models were developed following a consistent 9:1 training-to-test data split. The acoustic modeling process adhered to the GMM-HMM pipeline, beginning with monophone training and progressing through increasingly sophisticated triphone-based models. Word Error Rate (WER) served as the primary evaluation metric. Initial results from the monophone model yielded a WER of \textbf{43.64\%}. Subsequent enhancements using context-dependent triphones significantly reduced this to \textbf{6.75\%}. Incorporating speaker adaptation techniques through fMLLR in the SAT model further lowered the WER to \textbf{5.65\%}. The most accurate results were obtained using the triphone model with LDA and MLLT transformations, achieving a WER of \textbf{5.49\%}. These outcomes highlight the effectiveness of the GMM-HMM approach in modeling Akeanon speech and affirm the feasibility of deploying ASR technologies for underrepresented Philippine languages. This work establishes foundational linguistic resources and technological baselines for future initiatives in language documentation, revitalization, and accessibility.
%  Do not put citations or quotes in the abract.

% Suggested keywords based on ACM Computing Classification system can be found at \url{https://dl.acm.org/ccs/ccs_flat.cfm}

\begin{tabular}{lp{4.25in}}
\hspace{-0.5em}\textbf{Keywords:}\hspace{0.25em} & Language resources, Natural language processing (NLP), Speech recognition, Philippine languages, Aklan, Aklanon, Akeanon, Language corpus, Low-resource languages (LRL)\\
\end{tabular}
