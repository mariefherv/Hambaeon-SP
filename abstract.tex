%   Filename    : abstract.tex 
\begin{center}
\textbf{Abstract}
\end{center}
\setlength{\parindent}{0pt}
This study aimed to develop foundational resources and acoustic models to support automatic speech recognition (ASR) for the Akeanon language. A text corpus containing \textbf{25,800} verified Akeanon words was constructed, alongside additional translations of the Swadesh 207-word list and SIL International's word list for five major Akeanon dialects. Furthermore, a speech corpus consisting of \textbf{100} voice recordings, totaling approximately \textbf{16 hours} of speech data, was collected to provide training and evaluation material. Using the Kaldi toolkit, ASR models were trained following a fixed 9-1 data split approach, utilizing monophone and triphone acoustic modeling. Performance evaluation was conducted based on Word Error Rate (WER), revealing substantial improvements with more advanced models. The monophone model achieved a WER of \textbf{43.64\%}, while the triphone model significantly reduced the error to \textbf{6.75\%}. The best-performing model, triphone with LDA-MLTT, achieved a WER of \textbf{5.49\%}, demonstrating its effectiveness in recognizing Akeanon speech. These findings demonstrate that ASR technology can be successfully adapted for underrepresented Philippine languages, particularly those with complex phonetic structures. This study lays the groundwork for future research and applications aimed at enhancing language preservation and accessibility of the Akeanon language.
%  Do not put citations or quotes in the abract.

% Suggested keywords based on ACM Computing Classification system can be found at \url{https://dl.acm.org/ccs/ccs_flat.cfm}

\begin{tabular}{lp{4.25in}}
\hspace{-0.5em}\textbf{Keywords:}\hspace{0.25em} & Language resources, Natural language processing (NLP), Speech recognition, Philippine languages, Aklan, Aklanon, Akeanon, Language corpus, Low-resource languages (LRL)\\
\end{tabular}
