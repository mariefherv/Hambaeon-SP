%   Filename    : chapter_1.tex 
\chapter{Introduction}
\label{sec:researchdesc}    %labels help you reference sections of your document   %labels help you reference sections of your document

\section{Overview}
\label{sec:overview}

Speech-to-Text (STT) technology has rapidly evolved in recent years, driven by advancements in deep learning algorithms such as recurrent neural networks (RNNs) and convolutional neural networks (CNNs), which have significantly improved the accuracy of STT systems \cite{Televic:2024}. Open-source toolkits such as Kaldi have further accelerated research and development in this field by providing a flexible framework for building and training custom automatic speech recognition (ASR) models. ASR systems, which convert speech into text, have become essential components of various applications, from virtual assistants to transcription services \shortcite{Cerna:2023}. However, despite these advancements, only a few Philippine languages have been explored and integrated into this technology. This special problem focuses on one of the understudied \cite{Wellstood:2022} Central Philippine languages, Akeanon. 

Akeanon is an Austronesian language belonging to the Visayan subgroup \cite{Biray:2023}. With more than 130,000 households \cite{PSA:2023} speaking the language, Akeanon is primarily spoken in the province of Aklan, located in northwestern Panay. \citeA{Biray:2023} explains that the language has several dialects, each typically named after the town where it is spoken. These include Akeanon Buruangganon, Akeanon Nabasnon, Akeanon Bukidnon, and the standard Akeanon, which is spoken in most areas in Aklan including Kalibo, the provincial capital of Aklan. Additionally, the researchers will also explore Akeanon Malaynon for this study. For this special problem, the researchers will focus on developing the text and speech corpus for the Akeanon language, including all of its dialects.

Up to this date, no studies have been conducted that is directly related to Akeanon and speech recognition altogether. However, there exist similar studies in the context of speech recognition on other regional languages such as Bisaya in the study of \shortciteA{Cerna:2023}, Hiligaynon, studied by \shortciteA{Billones:2014} and \shortciteA{Panizales:2023}, and in the study of \shortciteA{Liaoetal:2019} for Bikol and Kapampangan. This special problem aims to bridge the gap in speech recognition for Akeanon starting with establishing a foundational speech corpus for the language, which can lay the groundwork for future research and applications. The corpus development will draw on methodologies from similar studies conducted for other regional languages such as the study of \shortciteA{Cerna:2023} and \shortciteA{Liaoetal:2019}, adapting them to meet the specific needs of Akeanon. In doing so, the project aims to bring Akeanon closer to digital integration, promoting inclusivity in speech recognition technology for Philippine languages. By bridging this gap, this special problem aspires to create a resource that can benefit future ASR developments, language preservation efforts, and the broader field of computational linguistics.

Creating a speech-to-text (STT) system for the Akeanon language not only fills the gap in representation for this regional language but also aids in its preservation and fosters digital inclusion. This specific project aims to establish a foundational corpus that effectively captures the distinct speech patterns and intricacies of Akeanon, while taking into account the language's unique phonetic and linguistic features. Utilizing the resources gathered for this research, the team will concentrate on developing a comprehensive text and speech corpus that can provide a basis for future speech recognition systems pertaining to the Akeanon language. The researchers will also build and train on the dataset of the constructed corpus using monophone and triphone models with Kaldi toolkit, to develop an ASR system that will provide initial speech recognition results for Akeanon. Finally, the study intends to investigate the challenges faced in developing speech models for languages with limited resources, offering valuable insights for the wider field of speech technology development.


\section{Problem Statement}
\label{sec:problemstatement}

Akeanon remains underrepresented in modern speech technologies. According to \citeA{Khan:2023}, in machine learning, natural language can be categorized into two categories: low-resource languages (LRLs) and high-resource languages (HRLs). Among these resources are (a) collections of text in different formats, such as research papers, journal articles, social media content, etc.; (b) lexical, syntactic, and semantic resources, such as dictionaries, bag of words, semantic databases, etc.; and (c) task-specific resources, such as annotated text, machine translation corpus, part-of-speech tags, etc.. HRLs e.g. English, French, Japanese, etc., are languages that are highly accessible and have many data resources that can be used for natural language processing (NLP). LRLs, on the other hand, are understudied and have few data resources that can be utilized for NLP.  Most regional languages in the Philippines are considered to be LRL, including the Akeanon language. \citeA{Alejan:2021} raised concerns on the Philippines' inclusion on a global list of the top ten "language hotspots", which means that many of its languages are disappearing faster than they are being completely documented. Their study noted the global rate of language extinction, which is one in every two weeks. They also projected that around half of the 6,000 languages will become extinct by the end of the century, to which most of them are indigenous languages. According to \shortciteA{Magueresse:2020}, a language supported by NLP techniques can help preserve it from extinction. It will also make the language more available and accessible in digital format, which offers significant commercial value, societal purpose, and applications in a variety of domains \cite{Tsvetkov:2017}.

This special problem aims to address the lack of resources, availability, and accessibility of the Akeanon language in, but not limited to, modern speech technologies by building and establishing a text and speech corpus for the language. Additionally, by developing an ASR model that is specific for Akeanon would lay the foundation for future research in speech-to-text, and other modern speech technologies for the language. Lastly, this special problem seeks to inspire innovation and drive similar efforts to preserve and develop accessible language technologies for other regional languages in the Philippines.

\section{Research Objectives}
\label{sec:researchobjectives}

\subsection{General Objective}
\label{sec:generalobjective}

The general objective of this study is to construct and establish a comprehensive text and speech corpus for the Akeanon language, which can serve as a foundation for future development of language technologies and automatic speech recognition (ASR) systems. Additionally, the study aims to design and implement an ASR system for the language using the Kaldi toolkit.

\subsection{Specific Objectives}
\label{sec:specificobjectives}

Specifically, the study aims to:
\begin{enumerate}
    \item develop an Akeanon text corpus by collecting existing language resources such as dictionaries, word lists, thesaurus, glossaries, and literary pieces (e.g., poems, fables, and tales) based in Akeanon and organizing them into an annotated dataset,
    \item build a speech corpus by recording native speakers and using pre-existing Akeanon audio resources which can be found online,
    \item validate the text and speech corpus with the assistance of linguistic experts and native speakers to ensure accuracy and reliability, and
    \item develop and evaluate an automatic speech recognition (ASR) model using monophone and triphone models and the Kaldi toolkit with the newly created Akeanon corpus.
 \end{enumerate}
 
\section{Scope and Limitations of the Research}
\label{sec:scopelimitations}

The system is specific to the Akeanon language, which is predominantly spoken in the province of Aklan. It is limited to the Akeanon languange, including its various dialects spoken in different parts of Aklan. The study is centered around gathering audio samples from native speakers of Akeanon to guarantee precision, though uniformity is not guaranteed since the study will include other variations or dialects of the Akeanon language. These include Akeanon Bukidnon, Akeanon Buruangganon, Akeanon Malaynon, and Akeanon Nabasnon, which can have different and unique phonetic and lexical traits. Non-digital resources will also be encoded and digitized to ensure accessibility and usability of the language in text format. Nevertheless, the model's effectiveness might be influenced by the scarce availability of Akeanon data, potentially affecting its wide-ranging applicability.


\section{Significance of the Research}
\label{sec:significance}

Akeanon language, like many indigenous languages in the Philippines, lacks representation in digital technologies. Establishing a foundational language corpora and creating an automatic speech recognition (ASR) system for Akeanon language will help contribute to the preservation of the language in digital format, establishing a resource that will support documentation and education initiatives in the future. The dataset and model produced in the study of Akeanon language can act as a basis for further and additional linguistic research.

Akeanon and its incorporation in speech recognition technology fosters digital inclusivity. This enables Akeanon speakers to engage with technology in their mother tongue highlighting the areas in education, communication, and public service where language barriers are almost present when accessing the said areas. Once a speech-to-text system for Akeanon has been established, mobile applications, AI assistants, translators, and other tools can embed the said technology to help enhance accessibility and boost engagement.

The challenge faced and lessons learned from this study will help contribute to addressing the lack of representation of low-resource language in AI technology, aligning with the need for inclusivity in language processing \cite{Poupard:2024}. This initiative will help in promoting linguistic diversity as well as safeguard cultural heritage through Akeanon speech recognition in technological advancement. \citeA{Poupard:2024} highlights that even minimal focus on languages with fewer resources can significantly influence their viability in an increasingly digital world where larger languages prevail.